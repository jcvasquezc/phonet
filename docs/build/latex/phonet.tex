%% Generated by Sphinx.
\def\sphinxdocclass{report}
\documentclass[letterpaper,10pt,english]{sphinxmanual}
\ifdefined\pdfpxdimen
   \let\sphinxpxdimen\pdfpxdimen\else\newdimen\sphinxpxdimen
\fi \sphinxpxdimen=.75bp\relax

\usepackage[utf8]{inputenc}
\ifdefined\DeclareUnicodeCharacter
 \ifdefined\DeclareUnicodeCharacterAsOptional\else
  \DeclareUnicodeCharacter{00A0}{\nobreakspace}
\fi\fi
\usepackage{cmap}
\usepackage[T1]{fontenc}
\usepackage{amsmath,amssymb,amstext}
\usepackage{babel}
\usepackage{times}
\usepackage[Bjarne]{fncychap}
\usepackage{longtable}
\usepackage{sphinx}

\usepackage{geometry}
\usepackage{multirow}
\usepackage{eqparbox}

% Include hyperref last.
\usepackage{hyperref}
% Fix anchor placement for figures with captions.
\usepackage{hypcap}% it must be loaded after hyperref.
% Set up styles of URL: it should be placed after hyperref.
\urlstyle{same}

\addto\captionsenglish{\renewcommand{\figurename}{Fig.}}
\addto\captionsenglish{\renewcommand{\tablename}{Table}}
\addto\captionsenglish{\renewcommand{\literalblockname}{Listing}}

\addto\extrasenglish{\def\pageautorefname{page}}

\setcounter{tocdepth}{2}



\title{phonet Documentation}
\date{Sep 30, 2019}
\release{0.1}
\author{Camilo Vasquez}
\newcommand{\sphinxlogo}{}
\renewcommand{\releasename}{Release}
\makeindex

\begin{document}

\maketitle
\sphinxtableofcontents
\phantomsection\label{\detokenize{index::doc}}


This toolkit compute posteriors probabilities of phonological classes from audio files for several groups of phonemes according to the mode and manner of articulation.

If you are not sure about what phonological classes are, have a look at this
\sphinxhref{http://research.cs.tamu.edu/prism/lectures/sp/l3.pdf}{Phonological classes tutorial}

The code for this project is available at \sphinxurl{https://github.com/jcvasquezc/phonet} .

The list of the phonological classes available and the phonemes that are activated for each phonological class are observed in the following Table

\noindent\begin{tabulary}{\linewidth}{|L|L|}
\hline
\sphinxstylethead{\relax 
Phonological class
\unskip}\relax &\sphinxstylethead{\relax 
Phonemes
\unskip}\relax \\
\hline
vocalic
&
/a/, /e/, /i/, /o/, /u/
\\
\hline
consonantal
&
/b/, /tS/, /d/, /f/, /g/, /x/, /k/, /l/, /ʎ/, /m/, /n/, /p/, /ɾ/, /r/, /s/, /t/
\\
\hline
back
&
/a/, /o/, /u/
\\
\hline
anterior
&
/e/, /i/
\\
\hline
open
&
/a/, /e/, /o/
\\
\hline
close
&
/i/, /u/
\\
\hline
nasal
&
/m/, /n/
\\
\hline
stop
&
/p/, /b/, /t/, /k/, /g/, /tS/, /d/
\\
\hline
continuant
&
/f/, /b/, /tS/, /d/, /s/, /g/, /ʎ/, /x/
\\
\hline
lateral
&
/l/
\\
\hline
flap
&
/ɾ/
\\
\hline
trill
&
/r/
\\
\hline
voiced
&
/a/, /e/, /i/, /o/, /u/, /b/, /d/, /l/, /m/, /n/, /r/, /g/, /ʎ/
\\
\hline
strident
&
/f/, /s/, /tS/
\\
\hline
labial
&
/m/, /p/, /b/, /f/
\\
\hline
dental
&
/t/, /d/
\\
\hline
velar
&
/k/, /g/, /x/
\\
\hline
pause
&
/sil/
\\
\hline\end{tabulary}



\chapter{Need Help?}
\label{\detokenize{help:welcome-to-phonet-s-documentation}}\label{\detokenize{help::doc}}\label{\detokenize{help:need-help}}
If you have trouble with Phonet, please write to Camilo Vasquez at: \sphinxhref{mailto:juan.vasquez@fau.de}{juan.vasquez@fau.de}

Supported features:
\begin{itemize}
\item {} 
\sphinxcode{phonet.get\_phon\_wav()} - Estimate the phonological classes for an audio file (.wav).

\item {} 
\sphinxcode{phonet.get\_phon\_path()} - Estimate the phonological classes for a folder that contains audio files (.wav) inside.

\item {} 
\sphinxcode{phonet.get\_posteriorgram()} - Plot the posteriorgram for an audio file (.wav).

\end{itemize}


\chapter{Installation}
\label{\detokenize{index:installation}}
From the source file:

\begin{sphinxVerbatim}[commandchars=\\\{\}]
\PYG{n}{git} \PYG{n}{clone} \PYG{n}{https}\PYG{p}{:}\PYG{o}{/}\PYG{o}{/}\PYG{n}{github}\PYG{o}{.}\PYG{n}{com}\PYG{o}{/}\PYG{n}{jcvasquezc}\PYG{o}{/}\PYG{n}{phonet}
\PYG{n}{cd} \PYG{n}{phonet}
\PYG{n}{python} \PYG{n}{setup}\PYG{o}{.}\PYG{n}{py} \PYG{n}{install}
\end{sphinxVerbatim}


\chapter{Methods}
\label{\detokenize{index:module-phonet}}\label{\detokenize{index:methods}}\index{phonet (module)}
Compute posteriors probabilities of phonological classes from audio files for several groups of phonemes according to the mode and manner of articulation.
Author: Camilo Vasquez-Correa 2019
\index{Phonet (class in phonet)}

\begin{fulllineitems}
\phantomsection\label{\detokenize{index:phonet.Phonet}}\pysiglinewithargsret{\sphinxstrong{class }\sphinxcode{phonet.}\sphinxbfcode{Phonet}}{\emph{phonological}}{}~\index{get\_feat() (phonet.Phonet method)}

\begin{fulllineitems}
\phantomsection\label{\detokenize{index:phonet.Phonet.get_feat}}\pysiglinewithargsret{\sphinxbfcode{get\_feat}}{\emph{signal}, \emph{fs}}{}
This method extracts log-Mel-filterbank energies used as inputs
of the model.
\begin{quote}\begin{description}
\item[{Parameters}] \leavevmode\begin{itemize}
\item {} 
\sphinxstyleliteralstrong{signal} -- the audio signal from which to compute features. Should be an N array.

\item {} 
\sphinxstyleliteralstrong{fs} -- the sample rate of the signal we are working with, in Hz.

\end{itemize}

\item[{Returns}] \leavevmode
A numpy array of size (NUMFRAMES by 33 log-Mel-filterbank energies) containing features. Each row holds 1 feature vector.

\end{description}\end{quote}

\end{fulllineitems}

\index{get\_phon\_path() (phonet.Phonet method)}

\begin{fulllineitems}
\phantomsection\label{\detokenize{index:phonet.Phonet.get_phon_path}}\pysiglinewithargsret{\sphinxbfcode{get\_phon\_path}}{\emph{audio\_path}, \emph{feat\_path}, \emph{plot\_flag=False}}{}
Estimate the phonological classes using the BGRU models for all the (.wav) audio files included inside a directory
\begin{quote}\begin{description}
\item[{Parameters}] \leavevmode\begin{itemize}
\item {} 
\sphinxstyleliteralstrong{audio\_path} -- directory with (.wav) audio files inside, sampled at 16 kHz

\item {} 
\sphinxstyleliteralstrong{feat\_path} -- directory were the computed phonological posteriros will be stores as a (.csv) file per (.wav) file from the input directory

\item {} 
\sphinxstyleliteralstrong{phonclass} -- phonological class to be evaluated (\sphinxquotedblleft{}consonantal\sphinxquotedblright{}, \sphinxquotedblleft{}back\sphinxquotedblright{}, \sphinxquotedblleft{}anterior\sphinxquotedblright{}, \sphinxquotedblleft{}open\sphinxquotedblright{}, \sphinxquotedblleft{}close\sphinxquotedblright{}, \sphinxquotedblleft{}nasal\sphinxquotedblright{}, \sphinxquotedblleft{}stop\sphinxquotedblright{},
\sphinxquotedblleft{}continuant\sphinxquotedblright{},  \sphinxquotedblleft{}lateral\sphinxquotedblright{}, \sphinxquotedblleft{}flap\sphinxquotedblright{}, \sphinxquotedblleft{}trill\sphinxquotedblright{}, \sphinxquotedblleft{}voice\sphinxquotedblright{}, \sphinxquotedblleft{}strident\sphinxquotedblright{},
\sphinxquotedblleft{}labial\sphinxquotedblright{}, \sphinxquotedblleft{}dental\sphinxquotedblright{}, \sphinxquotedblleft{}velar\sphinxquotedblright{}, \sphinxquotedblleft{}pause\sphinxquotedblright{}, \sphinxquotedblleft{}vocalic\sphinxquotedblright{}, \sphinxquotedblleft{}all\sphinxquotedblright{}).

\item {} 
\sphinxstyleliteralstrong{plot\_flag} -- True or False, whether you want plots of phonological classes or not

\end{itemize}

\item[{Returns}] \leavevmode
A directory with csv files created with the posterior probabilities for the phonological classes.

\end{description}\end{quote}

\begin{sphinxVerbatim}[commandchars=\\\{\}]
\PYG{g+gp}{\PYGZgt{}\PYGZgt{}\PYGZgt{} }\PYG{n}{directory}\PYG{o}{=}\PYG{n}{PATH}\PYG{o}{+}\PYG{l+s+s2}{\PYGZdq{}}\PYG{l+s+s2}{/phonclasses/}\PYG{l+s+s2}{\PYGZdq{}}
\PYG{g+gp}{\PYGZgt{}\PYGZgt{}\PYGZgt{} }\PYG{n}{phon}\PYG{o}{=}\PYG{n}{Phonet}\PYG{p}{(}\PYG{p}{[}\PYG{l+s+s2}{\PYGZdq{}}\PYG{l+s+s2}{vocalic}\PYG{l+s+s2}{\PYGZdq{}}\PYG{p}{,} \PYG{l+s+s2}{\PYGZdq{}}\PYG{l+s+s2}{strident}\PYG{l+s+s2}{\PYGZdq{}}\PYG{p}{,} \PYG{l+s+s2}{\PYGZdq{}}\PYG{l+s+s2}{nasal}\PYG{l+s+s2}{\PYGZdq{}}\PYG{p}{,} \PYG{l+s+s2}{\PYGZdq{}}\PYG{l+s+s2}{back}\PYG{l+s+s2}{\PYGZdq{}}\PYG{p}{,} \PYG{l+s+s2}{\PYGZdq{}}\PYG{l+s+s2}{stop}\PYG{l+s+s2}{\PYGZdq{}}\PYG{p}{,} \PYG{l+s+s2}{\PYGZdq{}}\PYG{l+s+s2}{pause}\PYG{l+s+s2}{\PYGZdq{}}\PYG{p}{]}\PYG{p}{)}
\PYG{g+gp}{\PYGZgt{}\PYGZgt{}\PYGZgt{} }\PYG{n}{phon}\PYG{o}{.}\PYG{n}{get\PYGZus{}phon\PYGZus{}path}\PYG{p}{(}\PYG{n}{PATH}\PYG{o}{+}\PYG{l+s+s2}{\PYGZdq{}}\PYG{l+s+s2}{/audios/}\PYG{l+s+s2}{\PYGZdq{}}\PYG{p}{,} \PYG{n}{PATH}\PYG{o}{+}\PYG{l+s+s2}{\PYGZdq{}}\PYG{l+s+s2}{/phonclasses2/}\PYG{l+s+s2}{\PYGZdq{}}\PYG{p}{)}
\end{sphinxVerbatim}

\end{fulllineitems}

\index{get\_phon\_wav() (phonet.Phonet method)}

\begin{fulllineitems}
\phantomsection\label{\detokenize{index:phonet.Phonet.get_phon_wav}}\pysiglinewithargsret{\sphinxbfcode{get\_phon\_wav}}{\emph{audio\_file}, \emph{feat\_file}, \emph{plot\_flag=True}}{}
Estimate the phonological classes using the BGRU models for an audio file (.wav)
\begin{quote}\begin{description}
\item[{Parameters}] \leavevmode\begin{itemize}
\item {} 
\sphinxstyleliteralstrong{audio\_file} -- file audio (.wav) sampled at 16 kHz

\item {} 
\sphinxstyleliteralstrong{feat\_file} -- file (.csv) to save the posteriors for the phonological classes

\item {} 
\sphinxstyleliteralstrong{phonclass} -- phonological class to be evaluated (\sphinxquotedblleft{}consonantal\sphinxquotedblright{}, \sphinxquotedblleft{}back\sphinxquotedblright{}, \sphinxquotedblleft{}anterior\sphinxquotedblright{}, \sphinxquotedblleft{}open\sphinxquotedblright{}, \sphinxquotedblleft{}close\sphinxquotedblright{}, \sphinxquotedblleft{}nasal\sphinxquotedblright{}, \sphinxquotedblleft{}stop\sphinxquotedblright{},
\sphinxquotedblleft{}continuant\sphinxquotedblright{},  \sphinxquotedblleft{}lateral\sphinxquotedblright{}, \sphinxquotedblleft{}flap\sphinxquotedblright{}, \sphinxquotedblleft{}trill\sphinxquotedblright{}, \sphinxquotedblleft{}voice\sphinxquotedblright{}, \sphinxquotedblleft{}strident\sphinxquotedblright{},
\sphinxquotedblleft{}labial\sphinxquotedblright{}, \sphinxquotedblleft{}dental\sphinxquotedblright{}, \sphinxquotedblleft{}velar\sphinxquotedblright{}, \sphinxquotedblleft{}pause\sphinxquotedblright{}, \sphinxquotedblleft{}vocalic\sphinxquotedblright{}, \sphinxquotedblleft{}all\sphinxquotedblright{}).

\item {} 
\sphinxstyleliteralstrong{plot\_flag} -- True or False, whether you want plots of phonological classes or not

\end{itemize}

\item[{Returns}] \leavevmode
A csv file created at FEAT\_FILE with the posterior probabilities for the phonological classes.

\end{description}\end{quote}

\begin{sphinxVerbatim}[commandchars=\\\{\}]
\PYG{g+gp}{\PYGZgt{}\PYGZgt{}\PYGZgt{} }\PYG{n}{phon}\PYG{o}{=}\PYG{n}{Phonet}\PYG{p}{(}\PYG{p}{[}\PYG{l+s+s2}{\PYGZdq{}}\PYG{l+s+s2}{stop}\PYG{l+s+s2}{\PYGZdq{}}\PYG{p}{]}\PYG{p}{)} \PYG{c+c1}{\PYGZsh{} get the \PYGZdq{}stop\PYGZdq{} phonological posterior from a single file}
\PYG{g+gp}{\PYGZgt{}\PYGZgt{}\PYGZgt{} }\PYG{n}{file\PYGZus{}audio}\PYG{o}{=}\PYG{n}{PATH}\PYG{o}{+}\PYG{l+s+s2}{\PYGZdq{}}\PYG{l+s+s2}{/audios/pataka.wav}\PYG{l+s+s2}{\PYGZdq{}}
\PYG{g+gp}{\PYGZgt{}\PYGZgt{}\PYGZgt{} }\PYG{n}{file\PYGZus{}feat}\PYG{o}{=}\PYG{n}{PATH}\PYG{o}{+}\PYG{l+s+s2}{\PYGZdq{}}\PYG{l+s+s2}{/phonclasses/pataka}\PYG{l+s+s2}{\PYGZdq{}}
\PYG{g+gp}{\PYGZgt{}\PYGZgt{}\PYGZgt{} }\PYG{n}{phon}\PYG{o}{.}\PYG{n}{get\PYGZus{}phon\PYGZus{}wav}\PYG{p}{(}\PYG{n}{file\PYGZus{}audio}\PYG{p}{,} \PYG{n}{file\PYGZus{}feat}\PYG{p}{,} \PYG{k+kc}{True}\PYG{p}{)}
\end{sphinxVerbatim}

\begin{sphinxVerbatim}[commandchars=\\\{\}]
\PYG{g+gp}{\PYGZgt{}\PYGZgt{}\PYGZgt{} }\PYG{n}{file\PYGZus{}audio}\PYG{o}{=}\PYG{n}{PATH}\PYG{o}{+}\PYG{l+s+s2}{\PYGZdq{}}\PYG{l+s+s2}{/audios/sentence.wav}\PYG{l+s+s2}{\PYGZdq{}}
\PYG{g+gp}{\PYGZgt{}\PYGZgt{}\PYGZgt{} }\PYG{n}{file\PYGZus{}feat}\PYG{o}{=}\PYG{n}{PATH}\PYG{o}{+}\PYG{l+s+s2}{\PYGZdq{}}\PYG{l+s+s2}{/phonclasses/sentence\PYGZus{}nasal}\PYG{l+s+s2}{\PYGZdq{}}
\PYG{g+gp}{\PYGZgt{}\PYGZgt{}\PYGZgt{} }\PYG{n}{phon}\PYG{o}{=}\PYG{n}{Phonet}\PYG{p}{(}\PYG{p}{[}\PYG{l+s+s2}{\PYGZdq{}}\PYG{l+s+s2}{nasal}\PYG{l+s+s2}{\PYGZdq{}}\PYG{p}{]}\PYG{p}{)} \PYG{c+c1}{\PYGZsh{} get the \PYGZdq{}nasal\PYGZdq{} phonological posterior from a single file}
\PYG{g+gp}{\PYGZgt{}\PYGZgt{}\PYGZgt{} }\PYG{n}{phon}\PYG{o}{.}\PYG{n}{get\PYGZus{}phon\PYGZus{}wav}\PYG{p}{(}\PYG{n}{file\PYGZus{}audio}\PYG{p}{,} \PYG{n}{file\PYGZus{}feat}\PYG{p}{,} \PYG{k+kc}{True}\PYG{p}{)}
\end{sphinxVerbatim}

\begin{sphinxVerbatim}[commandchars=\\\{\}]
\PYG{g+gp}{\PYGZgt{}\PYGZgt{}\PYGZgt{} }\PYG{n}{file\PYGZus{}audio}\PYG{o}{=}\PYG{n}{PATH}\PYG{o}{+}\PYG{l+s+s2}{\PYGZdq{}}\PYG{l+s+s2}{/audios/sentence.wav}\PYG{l+s+s2}{\PYGZdq{}}
\PYG{g+gp}{\PYGZgt{}\PYGZgt{}\PYGZgt{} }\PYG{n}{file\PYGZus{}feat}\PYG{o}{=}\PYG{n}{PATH}\PYG{o}{+}\PYG{l+s+s2}{\PYGZdq{}}\PYG{l+s+s2}{/phonclasses/sentence\PYGZus{}nasal}\PYG{l+s+s2}{\PYGZdq{}}
\PYG{g+gp}{\PYGZgt{}\PYGZgt{}\PYGZgt{} }\PYG{n}{phon}\PYG{o}{=}\PYG{n}{Phonet}\PYG{p}{(}\PYG{p}{[}\PYG{l+s+s2}{\PYGZdq{}}\PYG{l+s+s2}{strident}\PYG{l+s+s2}{\PYGZdq{}}\PYG{p}{,} \PYG{l+s+s2}{\PYGZdq{}}\PYG{l+s+s2}{nasal}\PYG{l+s+s2}{\PYGZdq{}}\PYG{p}{,} \PYG{l+s+s2}{\PYGZdq{}}\PYG{l+s+s2}{back}\PYG{l+s+s2}{\PYGZdq{}}\PYG{p}{]}\PYG{p}{)} \PYG{c+c1}{\PYGZsh{} get \PYGZdq{}strident, nasal, and back\PYGZdq{} phonological posterior from a single file}
\PYG{g+gp}{\PYGZgt{}\PYGZgt{}\PYGZgt{} }\PYG{n}{phon}\PYG{o}{.}\PYG{n}{get\PYGZus{}phon\PYGZus{}wav}\PYG{p}{(}\PYG{n}{file\PYGZus{}audio}\PYG{p}{,} \PYG{n}{file\PYGZus{}feat}\PYG{p}{,} \PYG{k+kc}{True}\PYG{p}{)}
\end{sphinxVerbatim}

\end{fulllineitems}

\index{get\_posteriorgram() (phonet.Phonet method)}

\begin{fulllineitems}
\phantomsection\label{\detokenize{index:phonet.Phonet.get_posteriorgram}}\pysiglinewithargsret{\sphinxbfcode{get\_posteriorgram}}{\emph{audio\_file}}{}
Estimate the posteriorgram for an audio file (.wav) sampled at 16kHz
\begin{quote}\begin{description}
\item[{Parameters}] \leavevmode
\sphinxstyleliteralstrong{audio\_file} -- file audio (.wav) sampled at 16 kHz

\item[{Returns}] \leavevmode
plot of the posteriorgram

\end{description}\end{quote}

\begin{sphinxVerbatim}[commandchars=\\\{\}]
\PYG{g+gp}{\PYGZgt{}\PYGZgt{}\PYGZgt{} }\PYG{n}{phon}\PYG{o}{=}\PYG{n}{Phonet}\PYG{p}{(}\PYG{p}{[}\PYG{l+s+s2}{\PYGZdq{}}\PYG{l+s+s2}{vocalic}\PYG{l+s+s2}{\PYGZdq{}}\PYG{p}{,} \PYG{l+s+s2}{\PYGZdq{}}\PYG{l+s+s2}{strident}\PYG{l+s+s2}{\PYGZdq{}}\PYG{p}{,} \PYG{l+s+s2}{\PYGZdq{}}\PYG{l+s+s2}{nasal}\PYG{l+s+s2}{\PYGZdq{}}\PYG{p}{,} \PYG{l+s+s2}{\PYGZdq{}}\PYG{l+s+s2}{back}\PYG{l+s+s2}{\PYGZdq{}}\PYG{p}{,} \PYG{l+s+s2}{\PYGZdq{}}\PYG{l+s+s2}{stop}\PYG{l+s+s2}{\PYGZdq{}}\PYG{p}{,} \PYG{l+s+s2}{\PYGZdq{}}\PYG{l+s+s2}{pause}\PYG{l+s+s2}{\PYGZdq{}}\PYG{p}{]}\PYG{p}{)}
\PYG{g+gp}{\PYGZgt{}\PYGZgt{}\PYGZgt{} }\PYG{n}{phon}\PYG{o}{.}\PYG{n}{get\PYGZus{}posteriorgram}\PYG{p}{(}\PYG{n}{file\PYGZus{}audio}\PYG{p}{)}
\end{sphinxVerbatim}

\end{fulllineitems}

\index{model() (phonet.Phonet method)}

\begin{fulllineitems}
\phantomsection\label{\detokenize{index:phonet.Phonet.model}}\pysiglinewithargsret{\sphinxbfcode{model}}{\emph{input\_size}, \emph{num\_labels=2}}{}
This is the architecture used for the estimation of the phonological classes
It consists of a 2 Bidirectional GRU layers, followed by a time-distributed dense layer
\begin{quote}\begin{description}
\item[{Parameters}] \leavevmode\begin{itemize}
\item {} 
\sphinxstyleliteralstrong{input\_size} -- size of input for the BGRU layers (number of features x sequence length).

\item {} 
\sphinxstyleliteralstrong{num\_labels} -- number of labels to be recogized by the DNN (2 for phonological posteriros and 21 for the phoneme recogizer).

\end{itemize}

\item[{Returns}] \leavevmode
A Keras model of a 2-layer BGRU neural network.

\end{description}\end{quote}

\end{fulllineitems}

\index{number2phoneme() (phonet.Phonet method)}

\begin{fulllineitems}
\phantomsection\label{\detokenize{index:phonet.Phonet.number2phoneme}}\pysiglinewithargsret{\sphinxbfcode{number2phoneme}}{\emph{seq}}{}
Converts the prediction of the neural network for phoneme recognition to a list of phonemes.
\begin{quote}\begin{description}
\item[{Parameters}] \leavevmode
\sphinxstyleliteralstrong{seq} -- sequence of integers obtained from the preiction of the neural network for phoneme recognition.

\item[{Returns}] \leavevmode
A list of strings of the phonemes recognized for each time-frame.

\end{description}\end{quote}

\end{fulllineitems}


\end{fulllineitems}



\chapter{Indices and tables}
\label{\detokenize{index:indices-and-tables}}\begin{itemize}
\item {} 
\DUrole{xref,std,std-ref}{genindex}

\item {} 
\DUrole{xref,std,std-ref}{modindex}

\item {} 
\DUrole{xref,std,std-ref}{search}

\end{itemize}


\chapter{Help}
\label{\detokenize{index:help}}
If you have trouble with Phonet, please write to Camilo Vasquez at: \sphinxhref{mailto:juan.vasquez@fau.de}{juan.vasquez@fau.de}


\renewcommand{\indexname}{Python Module Index}
\begin{sphinxtheindex}
\def\bigletter#1{{\Large\sffamily#1}\nopagebreak\vspace{1mm}}
\bigletter{p}
\item {\sphinxstyleindexentry{phonet}}\sphinxstyleindexpageref{index:\detokenize{module-phonet}}
\end{sphinxtheindex}

\renewcommand{\indexname}{Index}
\printindex
\end{document}